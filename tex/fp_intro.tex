\section{Einleitung}

Im Jahr 1911 entdeckte Ernest Rutherford den Atomkern, indem er die Streuung von $\alpha$-Teilchen an Goldfolie untersuchte \cite{rutherford}. Er stellte fest, dass die meisten $\alpha$-Teilchen die Goldfolie ungehindert passieren konnten und nur ca. $1$ von $100000$ $\alpha$-Teilchen abgelenkt oder reflektiert wurde. Dadurch schloss er darauf, dass die positive Ladung des Atoms in einem kompakten Atomkern in der Mitte des Atoms konzentriert und von einer Elektronenwolke umgeben ist.\\

Seit Rutherfords Experiment wurden viele weitere Experimente gemacht, um den Aufbau der Atomkerne zu untersuchen. Nach aktuellem Stand der Forschung bestehen diese aus Protonen und Neutronen, welche wiederum aus Quarks zusammengesetzt sind. Die Quarks und ihre Wechselwirkungen untereinander werden durch die QCD\footnote{\textbf{Q}uanten\textbf{C}hromo\textbf{D}ynamik} beschrieben.\\

Die Elektronen, welche den Atomkern umgeben zählen zu den sogenannten Leptonen. Die Quarks und Leptonen, die sogenannten Fermionen sind die Bausteine, aus denen die komplette Materie aufgebaut ist. Für die Struktur zwischen diesen Bausteinen und dafür, dass die Welt nicht auseinanderfällt, sorgen die Bosonen, welche die Wechselwirkungen zwischen den Fermionen übertragen.\\

Eines dieser Bosonen, das Z\textsuperscript0-Boson, wurde von 1989 bis 2000 am OPAL\footnote{\textbf Omni \textbf Purpose \textbf Apparatus at \textbf LEP}-Experiment am CERN\footnote{\textbf{C}onseil \textbf{E}uropéen pour la \textbf{R}echerche \textbf{N}ucléaire} untersucht. Aus den Daten dieses Experiments konnten Informationen über die elektroschwache Wechselwirkung und das Z\textsuperscript0-Boson gewonnen werden.\\

In diesem Versuch soll anhand von Originaldaten des OPAL-Experiments die Masse und Zerfallsbreite des Z\textsuperscript0-Bosons sowie die Partialbreiten der einzelnen Zerfallskanäle bestimmt werden, durch welche Erkenntnisse über geladene Leptonen, Neutrinos und Quarks gewonnen werden können. Außerdem wird der Weinbergwinkel bestimmt, welcher das Mischungsverhältnis zwischen schwacher und elektromagnetischer Wechselwirkung angibt.
\section{Das OPAL-Experiment}
%Saskia

OPAL (Omni Purpose Apparatus at LEP) war einer der vier größten Detektoren am Large Electron-Positron collider (LEP). Er wurde zusammen mit dem Beschleuniger im August 1989 in Betrieb genommen. Die Aufnahme von Daten mit OPAL endete am 2. November 2000. Der Detektor wurde im darauffolgenden Jahr abgebaut um Platz für den Large Hadron Collider (LHC) zu schaffen.\\


\graX{opal}{Aufbau des Opaldetektors}{Aufbau des Opaldetektors: Zwiebelartige Anordnung der Detektorschichten \label{Zwiebel} \footnotemark} \footnotetext{http://www.physicsmasterclasses.org/exercises/unischule/exp/opal-det.htm}

\subsection{Aufbau des Detektors}
Der OPAL-Detektor ist ca. $12\,\mathrm{m}$ hoch, breit und lang und besteht aus mehreren Detektorschichten, welche zwiebelartig um den Kollisionspunkt herum angeordnet sind (siehe Abbildung \ref{Zwiebel}). Im Inneren des Detektors befinden sich die Halbleiterdetektoren aus Silizium-Streifen, sowie die Vertex-Kammern, welche von der Jet-Kammer und den Z-Kammern umgeben sind. Diese Kammern dienen als Spurdetektoren um die Flugstrecke der geladenen Teilchen verfolgen zu können. Die Kammern befinden sich innerhalb eines Drucktanks, in welchem sich ein Gasgemisch aus Argon, Butan und Isobutan unter einem Druck $p=4\,\mathrm{bar}$ befindet. Der Drucktank ist von einem TOF\footnote{\textbf Time \textbf of \textbf Flight}-Detektor umgeben, welcher mit Szintillationszählern die Flugzeit der Teilchen bestimmt. Weiter außen befinden sich die elektromagnetischen und hadronischen Kalorimeter, welche die Energien der Teilchen messen. Ganz außen sind die Myonkammern, in welchen nur die Myonen registriert werden. \cite{anleitungalt}

\subsection{Identifikation von Teilchen und Klassifizierung der Ereignisse}
 
 Die unterschiedlichen Teilchen wechselwirken auf unterschiedliche Arten mit den Detektoren (siehe 
\ref{Teilchendetektor}). Im Folgenden wird erklärt, welche Teilchen in welchen Detektoren registriert werden und wie man die Teilchen auseinander halten kann.

\paragraph{Elektronen}
Elektronen sind im Spurdetektor sichtbar und lassen sich durch die deponierte Energie im elektromagnetischen Kalorimeter identifizieren.
\paragraph{Myonen}
Myonen werden als einzige Teilchen auch im Myonendetektor detektiert und sind damit gut identifizierbar.
\paragraph{Tauonen}
Tauonen sind wie auch Elektronen im Spurdetektor sichtbar, deponieren allerdings eine deutlich kleinere Energie im elektromagnetischen Kalorimeter.
\paragraph{Neutrinos}
Neutrinos werden in den Detektoren nicht beobachtet. Sie werden lediglich rückschließend durch fehlende Impuls- oder Energieanteile nachgewiesen.
\paragraph{Quarks und Hardonen}
Quarks können aufgrund ihrer Eigenschaften aus der QCD nicht alleine existieren, daher entstehen bei der Produktion von Quarks viele Hadronen, weshalb im Spurdetektor sehr vile Spuren zu sehen sind. Außerdem deponieren Hadronen als einzige Teilchen ihre Energie im hadronischen Kalorimeter.\\

Im folgenden Abschnitt werden diese Auswahlprinzipien noch einmal an den GROPE-Bildern (Detektor-Bilder) deutlich.
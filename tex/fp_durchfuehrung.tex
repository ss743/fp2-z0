\section{Das OPAL-Experiment}
%Saskia

OPAL (Omni Purpose Apparatus at LEP) war einer der vier größten Detektoren am Large Electron-Positron collider (LEP). Er wurde zusammen mit dem Beschleuniger im August 1989 in Betrieb genommen. Die Aufnahme von Daten mit OPAL endete am 2. November 2000. Der Detektor wurde im darauffolgenden Jahr abgebaut um Platz für den Large Hadron Collider (LHC) zu schaffen.
Der OPAL-Detektor war ca. $12\,\mathrm{m}$ hoch, breit und lang.\\


\graX{opal}{Aufbau des Opaldetektors}{Aufbau des Opaldetektors: Zwiebelartige Anrodnung der Detektorschichten \label{Zwiebel} \footnotemark} \footnotetext{http://www.physicsmasterclasses.org/exercises/unischule/exp/opal-det.htm}

\paragraph{Aufbau des Detektors}

\paragraph{Identifikation von Teilchen und Klassifizierung der Ereignisse}
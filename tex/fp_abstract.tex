\section*{Abstract}

Ziel dieses Versuches ist die Bestimmung der Masse und der Zerfallsbreite des Z\textsuperscript0-Bosons sowie der Partialbreiten der einzelnen Zerfallskanäle. Hierfür wird zunächst mit Monte-Carlo simulierten Daten gearbeitet und anschließend Originaldaten des OPAL-Experiments ausgewertet. Außerdem wurden Erkenntnise über die Leptonen, Neutrinos und Quarks gewonnen: Es wird die Leptonenuniversalität überprüft und die Anzahl der Neutrinofamilien bestimmt. Zusätzlich wird der Weinbergwinkel berechnet.

Die Ergebnisse der Zerfallsbreite des Z\textsuperscript0-Bosons sowie der Partialbreiten der einzelnen Zerfallskanäle sind in folgender Tabelle dargestellt:

\begin{table}[bh]
	\centering
	\begin{tabular}{c|cc}
		&gemessener Wert&theoretischer Wert\\\hline
		$\Gamma_l\,/\,\si{MeV}$&$247\pm2$&$250,24\pm0,03$\\
		$\Gamma_q\,/\,\si{MeV}$&$1788\pm18$&$1674,59\pm0,19$\\
		$\Gamma_\nu\,/\,\si{MeV}$&$500\pm20$&$497,64\pm0,03$\\
		$\Gamma_Z\,/\,\si{MeV}$&$2534\pm15$&$2422,5\pm0,2$\\
		$M_Z\,/\,\si{GeV}$&$91,183\pm0,006$&$91,188\pm0,002$
	\end{tabular}
	\caption{Ergebnisse für Zerfallsbreiten und Z\textsuperscript0-Masse}
	\label{tab:abstractergebnisse}
\end{table}

Hieraus wurden die Verzweigungsverhältnisse (Branching Ratio) BR der einzelnen Kanäle berechnet. Diese Werte sind in folgender Tabelle gelistet:

\begin{table}[bh]
	\centering
	\begin{tabular}{c|cc}
		&gemessener Wert&theoretischer Wert\\\hline
		$\mathrm{BR}_l\,/\,\si{\%}$&$9,7\pm1,0$&$10,3298\pm0,0013$\\
		$\mathrm{BR}_q\,/\,\si{\%}$&$19,7\pm0,9$&$20,543\pm0,002$\\
		$\mathrm{BR}_\nu\,/\,\si{\%}$&$70,6\pm0,8$&$69,123\pm0,010$
	\end{tabular}
	\caption{Ergebnisse für Verzweigungsverhältnisse}
	\label{tab:abstractbranch}
\end{table}

Aus dem Vergleich der gemessenen Zerfallsbreite des Neutrinokanals mit dem berechneten theoretischen Wert wurde die Anzahl der Neutrinofamilien bestimmt:

\begin{alignat*}{3}
&N_\nu&&=&\,3,01\pm0,14\text{,}\\
&N_\nu^\text{lit}&&=&3\text{.}
\end{alignat*} 

Der Weinbergwinkel wurde schließlich aus der Vorwärts-Rückwärts-Asymmetrie bestimmt:

\begin{alignat*}{4}
&\sin^2\theta_W&&=&0,21\pm0,02\text{,}\\
&\sin^2\theta_W^\text{lit}&&=&\,0,23116\pm0,00013\text{.}
\end{alignat*}
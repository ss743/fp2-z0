\section{Zusammenfassung und Diskussion}

Aus den experimentellen Daten des OPAL-Experiments haben wir die Wirkungsquerschnitte der verschiedenen Zerfallskanäle des Z\textsuperscript0-Bosons bei verschiedenen Schwerpunktsenergien bestimmt und daraus die partiellen und die gesamte Zerfallsbreite des Z\textsuperscript0-Bosons sowie seine Masse bestimmt. Dabei erhielten wir die in Tabelle \ref{tab:ergebnisse} beschriebenen und mit den in Abschnitt \ref{sec:partialbreiten} theoretisch berechneten Werten verglichenen Ergebnisse.\\

\begin{table}[bh]
	\centering
	\begin{tabular}{c|cc}
		&gemessener Wert&theoretischer Wert\\\hline
		$\Gamma_l\,/\,\si{MeV}$&$247\pm2$&$250,24\pm0,03$\\
		$\Gamma_q\,/\,\si{MeV}$&$1788\pm18$&$1674,59\pm0,19$\\
		$\Gamma_\nu\,/\,\si{MeV}$&$500\pm20$&$497,64\pm0,03$\\
		$\Gamma_Z\,/\,\si{MeV}$&$2534\pm15$&$2422,5\pm0,2$\\
		$M_Z\,/\,\si{GeV}$&$91,183\pm0,006$&$91,188\pm0,002$
	\end{tabular}
	\caption{Ergebnisse für Zerfallsbreiten und Z\textsuperscript0-Masse}
	\label{tab:ergebnisse}
\end{table}

Die Ergebnisse stimmen jeweils gut mit den theoretischen Werten überein, mit Ausnahme der hadronischen und der totalen Zerfallsbreite, welche höher als erwartet scheinen. Die Abweichung beträgt bei diesen beiden Werten ca. $5\,\sigma$. Möglicherweise ist diese Abweichung auf unsaubere Schnitte zurückzuführen. Außerdem waren die Schwerpunktsenergien des Detektors ohne Fehler angegeben. Wenn die Angaben der Energien nicht ganz den wahren Energien entsprechen, kann dies natürlich eine Verschiebung der Zerfallbreiten bewirken und führt zu einer zu geringen Fehlerabschätzung.\\

Außerdem wurden die Verzweigungsverhältnisse für geladene Leptonen, Hadronen und Neutrinos berechnet. Die Ergebnisse sind im Vergleich mit den in Abschnitt \ref{sec:partialbreiten} theoretisch berechneten Werten in Tabelle \ref{tab:branch} dargestellt.\\

\begin{table}[bh]
	\centering
	\begin{tabular}{c|cc}
		&gemessener Wert&theoretischer Wert\\\hline
		$\mathrm{BR}_l\,/\,\si{\%}$&$9,7\pm1,0$&$10,3298\pm0,0013$\\
		$\mathrm{BR}_q\,/\,\si{\%}$&$19,7\pm0,9$&$20,543\pm0,002$\\
		$\mathrm{BR}_\nu\,/\,\si{\%}$&$70,6\pm0,8$&$69,123\pm0,010$
	\end{tabular}
	\caption{Ergebnisse für Verzweigungsverhältnisse}
	\label{tab:branch}
\end{table}

Diese Werte stimmen innerhalb des Fehlers mit den theoretischen Verzweigungsverhältnissen überein.

Mit Hilfe der unsichtbaren Zerfallsbreite $\Gamma_\nu$ konnte die Anzahl der Neutrinogenerationen bestimmt werden (Literaturwert aus \cite{anleitungalt}):
\begin{alignat}{3}
	&N_\nu&&=&\,3,01\pm0,14\text{,}\\
	&N_\nu^\text{lit}&&=&3\text{.}
\end{alignat} 
Dies stimmt mit dem Literaturwert sehr gut überein.\\

Des Weiteren haben wir das Sinusquadrat des Weinbergwinkels $\sin^2\theta_W$ aus den Daten der Vorwärts-Rückwärts-Asymmetrie berechnet zu (Literaturwert aus \cite{nakamura}):
\begin{alignat}{4}
	&\sin^2\theta_W&&=&0,21\pm0,02\text{,}\\
	&\sin^2\theta_W^\text{lit}&&=&\,0,23116\pm0,00013\text{.}
\end{alignat}

Die Werte stimmen innerhalb von zwei Standardabweichungen überein, somit passt der gemessene Wert zur Theorie.
